\documentclass{res}
%\usepackage{helvetica}
%\usepackage{newcent}
\usepackage{amssymb}

\newsectionwidth{0pt}  % So the text is not indented under section headings
\usepackage{fancyhdr}  % use this package to get a 2 line header
\renewcommand{\headrulewidth}{0pt} % suppress line drawn by default by fancyhdr
\setlength{\headheight}{24pt} % allow room for 2-line header
\setlength{\headsep}{24pt}  % space between header and text
\setlength{\headheight}{24pt} % allow room for 2-line header
\pagestyle{fancy}     % set pagestyle for document
%\rhead{ {\it J. Wang}\\{\it p. \thepage} } % put text in header (right side)
%\cfoot{}                                     % the foot is empty
\topmargin=-0.5in % start text higher on the page

\begin{document}
% Center the name over the entire width of resume:
 \moveleft.5\hoffset\centerline{\large\bf Junwei WANG}
% Draw a horizontal line the whole width of resume:
 \moveleft\hoffset\vbox{\hrule width\resumewidth height 1pt}\smallskip
% address begins here
% Again, the address lines must be centered over entire width of resume:
\moveleft.5\hoffset\centerline{No. 1 Baidu Ke Ji Yuan,}
\moveleft.5\hoffset\centerline{10 Xi Bei Wang Dong Lu, Haidian Qu,}
\moveleft.5\hoffset\centerline{Beijing, China, 100094}
\moveleft.5\hoffset\centerline{i.junwei.wang@gmail.com}
\moveleft.5\hoffset\centerline{(+86) 185-1898-5736}
\newcommand\ignore[1]{}

\begin{resume}

\ignore{
\section{OBJECTIVE}
A position in the field of technology and computers with special
interests in cryptographic analysis, business applications programming,
information processing, management systems and so on. 
}

\vspace{0.2in}
\section{\centerline{EDUCATION}}
\vspace{15pt}
{\sl Master in Information and Computer Science}  \hfill
September 2013 - September 2014 \\
{\bf University of Luxembourg}, Luxembourg City, Luxembourg\\
Concentration: {\em Cryptography and Side-Channel Analysis}\\
%GPA: 17.08/20

{\sl Master of Computer Science and Technology}  \hfill
September 2012 - June 2015 \\
{\bf Shandong University}, Jinan City, Shandong Province, China \\
Concentration: {\em Cryptography and Security}

{\sl Bachelor of Software Engineer}  \hfill
September 2008 - June 2012 \\
{\bf Shandong University}, Jinan City, Shandong Province, China \\
%GPA: 89.35/100

\vspace{0.2in}
\section{\centerline{MASTER THESIS}}
\vspace{15pt}
{\sl Efficient Implementation of High-Order DPA Countermeasures for the AES
using the ARM NEON Instruction Set}, under the supervision of
Ass. Prof. Jean-S\'{e}bastien Coron


\vspace{0.2in}
\section{\centerline{WORKING EXPERIENCE}}
\vspace{15pt}
{\sl R\&D Engineer} \hfill July 2015 - now \\
{\bf Baidu Inc.}, Beijing, China \\
In charge of the workflow of creating knowledge graphs by processing of web
crawled data in the Searching Service Group, including,
\begin{itemize}  \itemsep -2pt
  \item Developing and maintaining a cleanse system (based on open source
    code) dealing with messy data: cleaning, transforming and extending it;
  \item Developing and maintaining the platform designing for R\&D engineers
    and outsourced staffs to build meaningful knowledge graphs inspired by
    structured or semi-structured data;
  \item Developing an adapter system to match the interfaces between the
    crawling system and its downstream systems, etc.
\end{itemize}

{\sl R\&D Engineer (Intern)} \hfill December 2014 - May 2015 \\
{\bf Eyespage}, Beijing, China
\begin{itemize}  \itemsep -2pt
  \item Designed and developed the API.
  \item Developed a spider to crawl data from Google Play Store by using the
    Scrapy framework.
  \item Operated and monitored with Elastic-Logstash-Kibana stack, Zabbix and
    so on.
  \item Co-designed the system architecture.
\end{itemize}

{\sl R\&D Engineer (Intern)} \hfill August 2011 - January 2012 \\
{\bf Baidu Inc.}, Beijing, China
\begin{itemize}  \itemsep -2pt
  \item Developed a ``user friendly" monitoring and warning system for
    online services of Baidu, mainly focusing on obtaining, processing and
    displaying data.
\end{itemize}



\ignore{
\section{COMPUTER \\ SKILLS}
{\sl Languages \& Software:} C, C++, PHP, Perl, Java, ARM NEON Assembly,
Python, Shell Script, Scala   Vim, MySQL, Matlab/Octave, LAMP and so on. \\
{\sl Operating Systems:} Linux.
}


\vspace{0.2in}
\section{\centerline{COMPUTER SKILLS}}
\vspace{15pt}
\begin{itemize}
  \item Solid skills in programming, especially in Java and Python programming
    languages, also using Shell Scripts, PHP, C, C\verb!++!, Perl, Scala, ARM
    NEON Assembly;
  \item Familiar with Unix-like operating systems;
  \item Familiar with distributed computing frameworks, e.g., Hadoop, HBase and
    Kafka;
  \item Familiar with popular relational database and NoSql databases, e.g.,
    MySql, Redis, Elasticsearch and so on.
\end{itemize}

\vspace{0.2in}
\section{\centerline{LANGUAGES}}
\vspace{15pt}
\begin{itemize}
  \item Chinese (mother tongue)
  \item English (intermediate)
\end{itemize}

\ignore{
\section{\centerline{HOBBIES}}
Reading, swimming and cycling.
}

\vspace{0.2in}
\section{\centerline{WEBSITES}}
\vspace{15pt}
\begin{itemize}
  \item {\bf Github} https://github.com/junwei-wang
  \item {\bf LinkedIn} https://www.linkedin.com/in/junweiwang
  %\item {\bf Blog} http://hotfixs.com
\end{itemize}

\vspace{0.2in}
\section{\centerline{PUBLICATIONS}}
\vspace{15pt}
[1] Junwei Wang, Praveen Kumar Vadnala, Johann Gro{\ss}sch{\"a}dl, and Qiuliang Xu.
Higher-Order Masking in Practice: A Vector Implementation of Masked AES for
ARM NEON. In Kaisa Nyberg, editor, {\em The Cryptographer’s Track at the RSA
Conference 2015. Proceedings}, volume 9048 of {\em Lecture Notes in Computer
Science}, pages 181{–}198. Springer, 2015.

\end{resume}

\end{document}

