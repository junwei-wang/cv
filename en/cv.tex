\documentclass{res}
\usepackage[dvipsnames]{xcolor}
\usepackage[sc]{mathpazo}
\linespread{1.05}         % Palladio needs more leading (space between lines)
\usepackage[T1]{fontenc}

\usepackage{amssymb}
\usepackage{hyperref}

\newsectionwidth{0pt}  % So the text is not indented under section headings
\usepackage{fancyhdr}  % use this package to get a 2 line header
\renewcommand{\headrulewidth}{0pt} % suppress line drawn by default by fancyhdr
\setlength{\headheight}{24pt} % allow room for 2-line header
\setlength{\headsep}{24pt}  % space between header and text
\setlength{\headheight}{24pt} % allow room for 2-line header
\pagestyle{fancy}     % set pagestyle for document
%\rhead{ {\it J. Wang}\\{\it p. \thepage} } % put text in header (right side)
%\cfoot{}                                     % the foot is empty
\topmargin=-0.5in % start text higher on the page

\def\ctitle#1#2#3{\emph{\textcolor{Maroon}{#1}}\hfill#2 - #3 \smallskip\\}

\begin{document}
\moveleft.5\hoffset\centerline{\Large\bf Junwei WANG}
\moveleft\hoffset\vbox{\hrule width\resumewidth height 1pt}\smallskip
\moveleft.5\hoffset\centerline{41 Boulevard des Capucines}
\moveleft.5\hoffset\centerline{75002 Paris, France}
\moveleft.5\hoffset\centerline{\href{mailto:junwei.wang@cryptoexperts.com}{junwei.wang@cryptoexperts.com}}
\moveleft.5\hoffset\centerline{(+33) 7 69 39 52 85}
\moveleft.5\hoffset\centerline{\href{https://junwei.co}{https://junwei.co}}
\newcommand\ignore[1]{}

\begin{resume}

\section{\large \centerline{EDUCATION}}
\bigskip
%%%%%%%%%%%%%%%%%%%%%%%%%%%%%%%%%
\ctitle{Ph.D. Candidate in Cryptography}{April 2017}{Now}
\textbf{CryptoExperts SAS}, Paris, France \\
\textbf{University of Luxembourg}, Esch-sur-Alzette, Luxembourg\\
\textbf{University Paris 8}, Saint-Denis, France
\smallskip \\
My research interests is white-box cryptography.
My thesis is under the supervisor of
Prof. \href{http://www.crypto-uni.lu/jscoron/index.html}{Jean-S\'{e}bastien Coron},
Prof. \href{http://www.math.univ-paris13.fr/~mesnager/}{Sihem Mesnager},
Dr. \href{https://scholar.google.com/citations?user=xwzhjfoAAAAJ&hl=en}{Pascal Paillier},
and Dr. \href{http://www.matthieurivain.com/}{Matthieu Rivain}.
I am an ECRYPT-NET fellow and receive funding from the European Union’s Horizon 2020 research and
innovation programme under the Marie Skłodowska-Curie grant agreement No. 643161.
\bigskip \\
%%%%%%%%%%%%%%%%%%%%%%%%%%%%%%%%%
\ctitle{Master in Information and Computer Science}{September 2013}{September 2014}
\textbf{University of Luxembourg}, Luxembourg City, Luxembourg
\smallskip\\
Thesis entitled \textsl{Efficient Implementation of High-Order DPA Countermeasures for the AES
  using the ARM NEON Instruction Set}, under the supervision of
Prof. \href{http://www.crypto-uni.lu/jscoron/index.html}{Jean-S\'{e}bastien Coron}.
\bigskip \\
%%%%%%%%%%%%%%%%%%%%%%%%%%%%%%%%%
\ctitle{Master of Computer Science and Technology}{September 2012}{June 2015}
\textbf{Shandong University}, Jinan, China
\bigskip\\
%%%%%%%%%%%%%%%%%%%%%%%%%%%%%%%%%
\ctitle{Bachelor of Software Engineer}{September 2008}{June 2012}
\textbf{Shandong University}, Jinan, China
\bigskip

\section{\large\centerline{WORKING EXPERIENCE}}
\bigskip
\ctitle{Research Intern}{April 2018}{July 2018}
\textbf{Riscure B.V.}, Delft, the Netherlands
\medskip\\
%%%%%%%%%%%%%%%%%%%%%%%%%%%%%%%%%%%%%%%%%%%%%%%%%%%%%%%%%%%%%%%%%%%%%%%
\ctitle{Senior Software Engineer}{July 2015}{April 2017}
\textbf{Baidu Inc.}, Beijing, China
\smallskip\\
I was at Knowledge Graph Department. My job was design and development of
systems for efficient production of knowledge data.
\medskip\\
%%%%%%%%%%%%%%%%%%%%%%%%%%%%%%%%%%%%%%%%%%%%%%%%%%%%%%%%%%%%%%%%%%%%%%%
\ignore{
In charge of the workflow of creating knowledge graphs by processing of web
crawled data in the Searching Service Group, including,
\begin{itemize}  \itemsep -2pt
  \item Developing and maintaining a cleanse system (based on open source
    code) dealing with messy data: cleaning, transforming and extending it;
  \item Developing and maintaining the platform designing for R\&D engineers
    and outsourced staffs to build meaningful knowledge graphs inspired by
    structured or semi-structured data;
  \item Developing an adapter system to match the interfaces between the
    crawling system and its downstream systems, etc.
  \end{itemize}
}
\ctitle{R\&D Engineer (Intern)}{December 2014}{May 2015}
\textbf{Eyespage}, Beijing, China \\\vspace{-3.5pt}
\begin{itemize}  \itemsep -1pt
  \item Designed and developed the API.
  \item Developed a spider to crawl data from Google Play Store by using the
    Scrapy framework.
  \item Operated and monitored with Elastic-Logstash-Kibana stack, Zabbix and
    so on.
  \item Co-designed the system architecture.
\end{itemize}
%\medskip
%%%%%%%%%%%%%%%%%%%%%%%%%%%%%%%%%%%%%%%%%%%%%%%%%%%%%%%%%%%%%%%%%%%%%%%
\ctitle{R\&D Engineer (Intern)}{August 2011}{January 2012}
\textbf{Baidu Inc.}, Beijing, China
\begin{itemize}  \itemsep -2pt
  \item Developed a ``user friendly" monitoring and warning system for
    online services of Baidu, mainly focusing on obtaining, processing and
    displaying data.
\end{itemize}


\ignore{
\section{COMPUTER \\ SKILLS}
{\sl Languages \& Software:} C, C++, PHP, Perl, Java, ARM NEON Assembly,
Python, Shell Script, Scala   Vim, MySQL, Matlab/Octave, LAMP and so on. \\
{\sl Operating Systems:} Linux.
}

\ignore{

\vspace{0.2in}
\section{\centerline{COMPUTER SKILLS}}
\vspace{15pt}
\begin{itemize}
  \item Solid skills in programming, especially in Java and Python programming
    languages, also using Shell Scripts, PHP, C, C\verb!++!, Perl, Scala, ARM
    NEON Assembly;
  \item Familiar with Unix-like operating systems;
  \item Familiar with distributed computing frameworks, e.g., Hadoop, HBase and
    Kafka;
  \item Familiar with popular relational database and NoSql databases, e.g.,
    MySql, Redis, Elasticsearch and so on.
  \end{itemize}
  }


\ignore{
\vspace{0.2in}
\section{\centerline{WEBSITES}}
\vspace{15pt}
\begin{itemize}
\item {\bf Homepage} http://junwei.co
\item {\bf Github} https://github.com/junwei-wang
\item {\bf LinkedIn} https://www.linkedin.com/in/junweiwang
\end{itemize}
}

\section{\large\centerline{PUBLICATIONS}}
\bigskip
[1] Junwei Wang, Praveen Kumar Vadnala, Johann Gro{\ss}sch{\"a}dl, and Qiuliang Xu.
Higher-Order Masking in Practice: A Vector Implementation of Masked AES for
ARM NEON. In Kaisa Nyberg, editor, {\em The Cryptographer’s Track at the RSA
Conference 2015. Proceedings}, volume 9048 of {\em Lecture Notes in Computer
Science}, pages 181{–}198. Springer, 2015.

\vspace{0.2in}
\section{\large \centerline{LANGUAGES}}
\vspace{15pt}
\begin{itemize}
  \item \emph{Chinese} (mother tongue) and \emph{English} (work proficiency)
\end{itemize}


\end{resume}

\end{document}


%%% Local Variables:
%%% mode: latex
%%% TeX-master: t
%%% End:
